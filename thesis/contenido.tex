\chapter{Contenidos del CD-ROM adjunto}\label{app:cdrom}
Por la parte interior de la tapa trasera se encuentra adjunto un CD-ROM con los
contenidos expuestos en esta memoria. Para acceder a este CD-ROM es necesario un
ordenador con lector de CD-ROM; no importando el sistema operativo que se use.

\section{Acceso al CD-ROM}
\subsubsection{Acceso desde sistemas windows}
\begin{enumerate}
\item Insertar el CD-ROM en el lector.
\item Acceder desde el icono \textit{Mi Pc} del escritorio o desde el
\textit{Explorador de ficheros} a la unidad correspondiente.
\item Si el sistema windows lo permite, la unidad aparecer� etiquetada como
\textit{PFC-TPCC}.
\end{enumerate}

\subsubsection{Acceso desde sistemas Mac OS 10.X}
\begin{enumerate}
\item Insertar el CD-ROM en el lector.
\item Acceder al \textit{finder} o al escritorio.
\item Si se accede desde el finder, en la parte superior izquierda se habr�
creado una nueva unidad que da acceso al CD-ROM.
\item Si se accede desde el escritorio, junto al icono de acceso al disco duro,
se habr� creado otro icono de acceso al CD-ROM.
\item Por �ltimo, si se accede desde consola, ir al directorio
\texttt{/Volumes/PFC-TPCC}
\end{enumerate}

\subsubsection{Acceso desde sistemas Linux/Unix}
\begin{enumerate}
\item Insertar el CD-ROM en el lector.
\item Si el sistema linux/unix dispone de sistema de montado autom�tico, en el
escritorio gr�fico o en los directorios \texttt{/mnt} o \texttt{/media}
aparecer� el CD-ROM accesible.
\item Si el sistema no dispone de un sistema de montado autom�tico, habr� que
utilizar la orden mount para acceder a los contenidos del CD-ROM
\item En cualquier caso, teclear \texttt{mount} para conocer el estado de los
dispositivos de almacenamiento montados en el sistema.
\item M�s informaci�n sobre la orden mount tecleando \texttt{man mount}
\end{enumerate}

\section{Listado del contenido}
Partiendo del directorio ra�z del CD-ROM, encontramos:
\begin{itemize}
\item \texttt{/memoria.pdf} Una copia de esta memoria en formato PDF.
\item \texttt{/fuentes/} Directorio donde se almacena el c�digo fuente.
	\begin{itemize}
	\item \texttt{tpcc.tar.gz} C�digo fuente del benchmark TPC-C.
	\end{itemize}
\item \texttt{/original/} Directorio donde se almacenan los originales en formato
\LaTeX de esta memoria.
	\begin{itemize}
        \item \texttt{memoria.tar.gz} Originales en formato \LaTeX.
	\end{itemize}
\end{itemize}

\subsubsection{Contenido del fichero \texttt{tpcc.tar.gz}}
Para descomprimir y desempaquetar este fichero desde un terminal unix/linux,
teclear para descomprimir en el directorio actual:\\
\texttt{tar -xzvf /ruta.al.fichero/tpcc.tar.gz}

Se obtendr� un directorio \texttt{tpcc/} donde se encuentra el c�digo fuente,
para m�s informaci�n sobre c�mo compilar dicho c�digo fuente, ver el manual de
la aplicaci�n (pag. \pageref{app:manual}).

\subsubsection{Contenido del fichero \texttt{memoria.tar.gz}}
Para descomprimir y desempaquetar este fichero desde un terminal unix/linux, 
teclear para descomprimir en el directorio actual:\\
\texttt{tar -xzvf /ruta.al.fichero/memoria.tar.gz}

Esta memoria ha sido escrita en \LaTeX, y adem�s ha sido orientada hacia la
obtenci�n de un documento PDF, por lo que es imposible utilizar la orden
\texttt{latex} para generar un fichero PS.

\paragraph{Generaci�n de la memoria}
Para generar un fichero PDF hay que teclear, en el directorio donde se haya
desempaquetado la memoria:
\begin{verbatim}
]$ pdflatex memoria
]$ bibtex memoria1
]$ bibtex memoria2
]$ pdflatex memoria
]$ pdflatex memoria
\end{verbatim}

\paragraph{Manejo de ficheros PDF}
La memoria se entrega en formato PDF (\textit{Portable Document Format}); y para
acceder al contenido se necesita un visor de ficheros PDF. Una lista de las
diferentes posibilidades seg�n el sistema operativo, es la siguiente:
\begin{itemize}
\item \textit{Adobe Acrobat Reader}: para los sistemas operativos Microsoft Windows, IBM OS/2
Warp, Palm OS, Pocket PC, Symbian OS, Linux, Sun Solaris, IBM Aix, HP-UX y Mac OS
7.x -- 10.x . Disponible en:
\url{http://www.adobe.es/products/acrobat/readstep2.html}.
\item \textit{xpdf}: para los sistemas operativos con el sistema de ventanas X
Window. Disponible en: \url{http://www.foolabs.com/xpdf/}.
\item \textit{kpdf}: forma parte del entorno de escritorio KDE. Disponible en
\url{http://www.kde.org/}; m�s informaci�n en: \url{http://kpdf.kde.org/}.
\item \textit{Vista Previa}: disponible en el sistema operativo Mac OS 10.x .
M�s informaci�n en: \url{http://www.apple.com/es/macosx/features/pdf/}.
\end{itemize}

